\documentclass{article}
\usepackage{hyperref}
\usepackage[utf8]{inputenc}
\usepackage[T1]{fontenc}
\begin{document}

  \section{Fonctionnement interne}
  \subsection{Analyse syntaxique}
  L'analyse syntaxique se fait en utilisant les outils Menhir et Ocamllex sans difficulté particulière.
  Elle produit un arbre de syntaxe abstraite non typé et conforme à la grammaire spécifiée par le sujet.
  \subsection{Typage}
  Alors que l'analyse syntaxique effectue seulement une vérification, le typage doit ajouter de l'information à l'AST défini auparavant.
  Le programme crée donc pour cela un environnement qui contient:
  \begin{itemize}
    \item \texttt{vars}: qui à un identifiant associe le type de l'identifiant (au plus grand niveau de déclaration possible). Cet identifiant peut correspondre à une déclaration de fonction, de variable ou de type.
    \item \texttt{types}: qui à un couple (niveau,identifiant) renvoie le type déclaré par l'identifiant au niveau donné.
    \item \texttt{idents}: liste des identifiants d'un niveau de déclaration et permet de vérifier que l'on n'utilise pas deux fois le même identifiant dans un niveau de déclarations.
    \item \texttt{records\_to\_check}: liste des enregistrements qui ont été déclarés mais non définis et où il faut donc vérifier la définition avant de passer à un autre niveau de déclarations.
  \end{itemize}
  Le typage consiste donc à:
  \begin{itemize}
    \item pour les expressions: inférer les types intermédiaires.
    \item pour les instructions: vérifier les correspondances de types.
    \item pour les déclarations: déclarer les types, les variables et les fonctions.
  \end{itemize}

  Les informations sauvegardées dans l'AST décoré envoyé en retour sont donc: les types intermédiaires pour les expressions, les types de retour pour les instructions.
  Pour les déclarations, l'important est de sauvegarder l'environnement d'exécution pour chaque déclaration de fonction.
  Ainsi lorsque qu'une fonction est déclarée, on récupère l'environnement induit par les déclarations dans la fonction et on le sauvegarde pour l'exécution des instructions.
  \section{Choix et problèmes rencontrés}
  \subsection{Problèmes de visualisation}
  Il est intéressant de pouvoir visualiser le comportement de l'analyse syntaxique et du typage de manière simple et intuitive.
  C'est pourquoi j'ai développé un petit outil dans le fichier \texttt{ast\_printer.ml} qui permet de faire une représentation graphique de l'arbre de syntaxe abstraite typé. \\
  Pour obtenir cette version modifiée du programme il suffit d'utiliser \texttt{make dot} en ayant le package ocamldot (\url{http://zoggy.github.io/ocamldot/}) préalablement installé.
  Éxecuter \texttt{adac} sur un fichier source génère alors, si la syntaxe est correcte, deux fichiers \texttt{graph.dot} et \texttt{graph.svg} correspondant à l'arbre de syntaxe abstraite typé.
  \subsection{Problèmes d'effacement des variables}
  La définition des règles de jugement des types n'empêche pas la définition à deux niveaux différents de variables de même nom mais de types différents.
  Ce n'est pas un problème: en effet seule la déclaration de niveau maximal compte pour la suite des déclarations et instructions.
  Le problème vient lors de la déclaration de variables de types. En effet si on traite les types comme des variables simples alors des problèmes arrivent:
  par exemple si l'on déclare une variable \texttt{x} d'un type \texttt{access R}, et que la déclaration du type \texttt{R} est ensuite écrasée par la déclaration d'un nouveau type \texttt{R}, alors on ne peut plus accéder aux véritables champs de \texttt{x}.
  Il faut donc sauvegarder l'ensemble des déclarations de types: c'est l'enjeu de la variable \texttt{type} de l'environnement de typage.
\end{document}
